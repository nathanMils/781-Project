% It is an example file showing how to use the 'sigkddExp.cls' 
% LaTeX2e document class file for submissions to sigkdd explorations.
% It is an example which *does* use the .bib file (from which the .bbl file
% is produced).
% REMEMBER HOWEVER: After having produced the .bbl file,
% and prior to final submission,
% you need to 'insert'  your .bbl file into your source .tex file so as to provide
% ONE 'self-contained' source file.
%
% Questions regarding SIGS should be sent to
% Adrienne Griscti ---> griscti@acm.org
%
% Questions/suggestions regarding the guidelines, .tex and .cls files, etc. to
% Gerald Murray ---> murray@acm.org
%

\documentclass{sigkddExp}

\usepackage{titlesec}
\titlespacing*{\section}{0pt}{1em}{0.5em} % Adjusts space before and after section
\titlespacing*{\subsection}{0pt}{0.75em}{0.5em}

\begin{document}
%
% --- Author Metadata here ---
% -- Can be completely blank or contain 'commented' information like this...
%\conferenceinfo{WOODSTOCK}{'97 El Paso, Texas USA} % If you happen to know the conference location etc.
%\CopyrightYear{2001} % Allows a non-default  copyright year  to be 'entered' - IF NEED BE.
%\crdata{0-12345-67-8/90/01}  % Allows non-default copyright data to be 'entered' - IF NEED BE.
% --- End of author Metadata ---

\title{Classifying Phishing Websites Using Machine Learning Techniques}
%\subtitle{[Extended Abstract]
% You need the command \numberofauthors to handle the "boxing"
% and alignment of the authors under the title, and to add
% a section for authors number 4 through n.
%
% Up to the first three authors are aligned under the title;
% use the \alignauthor commands below to handle those names
% and affiliations. Add names, affiliations, addresses for
% additional authors as the argument to \additionalauthors;
% these will be set for you without further effort on your
% part as the last section in the body of your article BEFORE
% References or any Appendices.

\numberofauthors{1}
%
% You can go ahead and credit authors number 4+ here;
% their names will appear in a section called
% "Additional Authors" just before the Appendices
% (if there are any) or Bibliography (if there
% aren't)

% Put no more than the first THREE authors in the \author command
%%You are free to format the authors in alternate ways if you have more 
%%than three authors.

\author{
%
% The command \alignauthor (no curly braces needed) should
% precede each author name, affiliation/snail-mail address and
% e-mail address. Additionally, tag each line of
% affiliation/address with \affaddr, and tag the
%% e-mail address with \email.
\alignauthor Nathan Opperman \\
       \affaddr{University of Pretoria}\\
       \affaddr{u21553832}\\
       \email{u21553832@tuks.co.za}
}

\date{30 October 2024}
\maketitle
\begin{abstract}
Text here...
\end{abstract}

\section{Introduction}
Text here...

\subsection{Problem Statement}
\label{prob_statement}
Phishing is the malicious practice where attackers impersonate organizations, such as banks or retailers, through emails or messages, in an attempt to trick individuals into revealing sensitive information. Phishing websites, which mimic legitimate sites, play a crucial role in these attacks by luring users to enter their private information. It is quite clear that phishing attacks pose a significant threat and so there is a need for effective detection methods to identify and mitigate these threats. 

\subsection{Research Questions}

\subsubsection{What are the key features that differentiate phishing websites from legitimate websites?}
\label{rq_1}
Text...
\subsubsection{What level of computational efficiency (processing time, resource usage) is required for practical phishing detection systems without compromising accuracy?}
\label{rq_2}
Text...
\subsubsection{What ML method will be best suited with efficiency and usability in mind?}
\label{rq_3}
Text...
\subsection{Research Objectives}
\label{research_objs}
With this project, I intend to develop and compare ML models, capable of accurately (and efficiently - important) detecting these phishing websites to enhance cybersecurity. To evaluate these models I intend to develop a novel prototype detection mechanism that can detect phishing websites in near real-time.
\iffalse
\subsection{Aims and Limitations}
\label{aims_and_lims}
The scope of this
\fi
\subsection{Methodology}
Text here...

\subsubsection{Literature Survey}
Text here...
\subsubsection{Exploratory Data Analysis}
Text here...
\subsubsection{Data Collection}
Text here...
\subsubsection{Feature Selection}
Text here...
\subsubsection{Hyperparameter Optimization}
Text here...
\subsubsection{Classification}
Text here...
\subsubsection{Prototype Detection Mechanism}
Text here...
\subsubsection{Experimentation}
Text here...
\subsubsection{Critical Evaluation}
Text here...
\cite{bowman:reasoning}

\section{Literature Survey}
Text here...
\subsection{Phishing}
A Phishing attack is described as a deceiving practice where the main idea of the attack is to impersonate a trustworthy entity in order to obtain unauthorized access to sensitive information for malicious \cite{10.5555/1071752.1071800, Ramzan2010}.

\vspace{1em}

The book \cite{Ramzan2010} further characterize phishing attacks using these three characteristics:

\begin{itemize}
	\item \textbf{A brand must be spoofed/impersonated}
	\item \textbf{A website must be involved}
	\item \textbf{Sensitive Information must be solicited}
\end{itemize}

The 2nd of which brings us to our next section.
\subsubsection{Phishing Websites}
Text here...
\subsubsection{Phishing Indicators}
Text here...
\subsection{Usability of ML for Security}
Text...
\subsection{Trade-offs between Accuracy and Efficiency}
Text...
\subsection{Existing applications of ML in Phishing Detection}
Text here...
\subsection{Decision Trees}
Text here...
\subsection{XGBoost}
Text here...
\subsection{SVM}
Text here...

\section{Data Collection}
For this project, two sources of data were required, firstly we utilized the popular \textbf{Phishing Websites} dataset
\section{Exploratory Data Analysis}
\label{eda}
In this section we discuss and describe the dataset \textbf{Phishing Websites} and the insights gained though visualizations and analysis.
\subsection{Description}
\label{descprition}
The Phishing Websites dataset contains a total of 11055 instances with 30 categorical ternary attributes derived from phishing indicators discussed and presented in \cite{6470857} as part of an effort to develop a group of features that are sound and effective in predicting phishing websites. There are a total of 12 Address Bar features, 6 Abnormal features, 5 HTML/JS features, 7 domain features, and a target feature that indicates if the website is a phishing website or not. Each attribute presents either one of three values: 1 indicating legitimate, 0 indicating suspicious, and -1 indicating phishing. Well some features only present 2 of the given values, what they represent is consistent.


 
\section{Data Mining Methods and Implementation}
Text here...

\section{Results}
Text here...

\section{Evaluation}
Text here...

\section{Conclusion}
Text here...

%
% The following two commands are all you need in the
% initial runs of your .tex file to
% produce the bibliography for the citations in your paper.
\bibliographystyle{abbrv}
\bibliography{sigproc}  % sigproc.bib is the name of the 

\appendix
\section{Phishing Website Features}


\end{document}
